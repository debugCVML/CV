% !TEX TS-program = xelatex
% !TEX encoding = UTF-8 Unicode
% !Mode:: "TeX:UTF-8"

\documentclass{resume}
\usepackage{zh_CN-Adobefonts_external} % Simplified Chinese Support using external fonts (./fonts/zh_CN-Adobe/)
%\usepackage{zh_CN-Adobefonts_internal} % Simplified Chinese Support using system fonts
\usepackage{linespacing_fix} % disable extra space before next section

\begin{document}
\pagenumbering{gobble} % suppress displaying page number

\name{张亚南}

% {E-mail}{mobilephone}{homepage}
\contactInfo{\textbf{zhangxiaoya1989@gmail.com}}{\textbf{(+86) 188-130-18216}}{\textbf{应聘职位:C++研发工程师}}
% {E-mail}{mobilephone}
%\basicContactInfo{\textbf{zhangxiaoya1989@gmail.com}}{\textbf{(+86) 188-130-18216}}
%\section{\faGraduationCap\  基本信息}
%\begin{itemize}[parsep=0.5ex]
%%\item \textbf{个人信息:} 男/硕士/智能信息技术北京市重点实验室
%\item \textbf{研究方向:} 图像与视频处理( Visual Tracking, Image Super Resolution),机器学习
%\item \textbf{期望职位:C++研发工程师}
\section{\faGraduationCap\  教育背景}
\datedsubsection{\textbf{北京理工大学} }{硕士(top20\%)}
%\textit{在读硕士研究生}\ 计算机科学与技术, 预计 2016 年 3 月毕业
\datedsubsection{\textbf{青岛科技大学} }{学士(top  3\%)}
%\textit{学士}\ 计算机科学与技术

\section{\faCogs\ IT 技能}
% increase linespacing [parsep=0.5ex]
\begin{itemize}[parsep=0.5ex]
  \item 编程语言: C++ > C > Java > Python > Lisp
%  \item 开发平台: Linux / Windows
  \item 深度技能: 图像处理/神经网络算法/C++对象模型/Redis框架/Linux和Windows平台研发
  %\item 开发工具: QtCreator / Visual Studio / Vim / Sublime / Codeblocks / CMake / Eclipse
  \item 其他技能: MySql / Oracle / $Git\& Github$ / Visio / Latex / Markdown / 英语 - 读写熟练
\end{itemize}

\section{\faUsers\ 科研经历}

\datedsubsection{\textbf{基于字典学习的鲁棒性视觉跟踪算法的研究(国家自然科学基金)}}{2014年06月 -- 至今}
%\role{国家自然科学基金}{研究生课题}
\begin{onehalfspacing}
基于机器学习方法,改进现有的基于字典学习目标跟踪方法,提出并实现鲁棒性更强的目标跟踪算法
\begin{itemize}
  \item 研究基于机器学习的目标跟踪算法、贝叶斯推理和粒子滤波理论
  \item 实现基于修正字典增强鲁棒性的目标跟踪算法、较基础算法准确度提升20\%
%  \item xxx (尽量使用量化的客观结果)
\end{itemize}
\end{onehalfspacing}

\datedsubsection{\textbf{基于BP神经网络的图像超分辨率算法的研究(研究生课题)}}{2013年09月 -- 2014年06月}
%\role{国家自然科学基金}{研究生课题}
\begin{onehalfspacing}
研究BP神经网络,以及BP神经网络在单帧图像超分辨率算法的应用
\begin{itemize}
  \item 研究学习BP神经网络和卷积神经网络、使用Matlab语言实现BP神经网络模拟算法
%  \item 提出并实现基于修正字典增强鲁棒性的目标跟踪算法、发表一篇学术论文
%  \item xxx (尽量使用量化的客观结果)
\end{itemize}
\end{onehalfspacing}

\section{\faUsers\ 实习/项目经历}

\datedsubsection{\textbf{三星通信技术研究院(云计算实验室-实习)} 北京}{2015年7月 -- 至今}
%\role{企业实习}{云计算实验室}
S-Voice持续化集成解决方案研发
\begin{itemize}
  \item 学习使用Jenkins部署持续化集成开发环境,学习部署Docker
%  \item 阅读并总结云计算专利和论文
\end{itemize}

\datedsubsection{\textbf{航天某部门仿真系统研发(V1.0-V2.0)}}{2014年12月 -- 2015年07月}
\role{C++ / MFC, Windows}{实验室合作项目,三人小组研发}
\begin{onehalfspacing}
根据多种传感器观测的数据,以及目标的特性数据,计算某种飞行目标的轨迹的仿真系统研发
\begin{itemize}
  \item 独立研发特性数据、传感器观测数据的数据库管理系统、内存数据库缓存
 % \item 开发内存数据库模块,加快仿真计算速度
%  \item xxx (尽量使用量化的客观结果)
\end{itemize}
\end{onehalfspacing}

\datedsubsection{\textbf{航空某部门图像标注软件研发}}{2015年01月 -- 2015年03月}
\role{C++ / Qt / OpenCV, Linux}{外援项目,两人小组研发}
\begin{onehalfspacing}
特性图片分割与图片标注,以及基于MySql的数据库管理系统综合软件研发
\begin{itemize}
  \item 独立研发数据库管理模块,复杂数据库操作、基于Qt的UI设计,绘图模块,获取图像标注区域
%  \item
%  \item xxx (尽量使用量化的客观结果)
\end{itemize}
\end{onehalfspacing}

%\datedsubsection{\textbf{青岛某公司工业防火墙实时监控系统客户端研发}}{2011年05月 -- 2011年10月}
%\role{C++ / Qt, Windows}{实验室合作项目,三人小组研发}
%\begin{onehalfspacing}
%基于拓扑结构实时监控所有设备防火墙工作状况和防火墙日志的数据库管理系统综合软件研发
%\begin{itemize}
%  \item 独立研发数据库管理模块
%  \item 完成数据多种格式显示模块
%  \item 基于Qt的UI设计,绘图模块,获取图像标注区域
%%  \item xxx (尽量使用量化的客观结果)
%\end{itemize}
%\end{onehalfspacing}

%\datedsubsection{\textbf{分布式科学上网姿势}}{2014年6月 -- 至今}
%\role{Golang, Linux}{个人项目,和富帅糕合作开发}
%\begin{onehalfspacing}
%分布式负载均衡科学上网姿势, https://github.com/cyfdecyf/cow
%\begin{itemize}
%  \item 修复了连接未正常关闭导致文件描述符耗尽的 bug
%  \item 使用Chord 哈希 URL, 实现稳定可靠地分流
%  \item xxx (尽量使用量化的客观结果)
%\end{itemize}
%\end{onehalfspacing}

%\datedsubsection{\textbf{\LaTeX\ 简历模板}}{2015 年5月 -- 至今}
%\role{\LaTeX, Python}{个人项目}
%\begin{onehalfspacing}
%优雅的 \LaTeX\ 简历模板, https://github.com/billryan/resume
%\begin{itemize}
%  \item 容易定制和扩展
%  \item 完善的 Unicode 字体支持,使用 \XeLaTeX\ 编译
%  \item 支持 FontAwesome 4.3.0
%\end{itemize}
%\end{onehalfspacing}



\section{\faHeartO\ 获奖情况}
\begin{itemize}[parsep=0.5ex]
  \item 荣誉称号: 山东省优秀毕业生、青岛科技大学优秀毕业生
  \item 奖学金: 国家励志奖学金(2 次)、一等奖学金(4 次)、二等奖学金(3 次)
\end{itemize}
%\datedline{荣誉称号:}{\textit{第一名}, xxx 比赛}{2013 年6 月}
%\datedline{其他奖项}{2015}

\section{\faInfo\ 其他}
% increase linespacing [parsep=0.5ex]
\begin{itemize}[parsep=0.5ex]
  \item 技术博客: http://zhangxiaoya.github.io
  \item GitHub: https://github.com/zhangxiaoya
%  \item 语言: 英语 - 读写熟练
  \item 个人总结:善于钻研,为人真诚,责任心强
\end{itemize}

\end{document}
