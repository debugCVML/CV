% !TEX TS-program = xelatex
% !TEX encoding = UTF-8 Unicode
% !Mode:: "TeX:UTF-8"

\documentclass{resume}
\usepackage{zh_CN-Adobefonts_external} % Simplified Chinese Support using external fonts (./fonts/zh_CN-Adobe/)
%\usepackage{zh_CN-Adobefonts_internal} % Simplified Chinese Support using system fonts
\usepackage{linespacing_fix} % disable extra space before next section

\begin{document}
\pagenumbering{gobble} % suppress displaying page number

\name{张亚南}

% {E-mail}{mobilephone}{homepage}
\contactInfo{\textbf{ynzhang.vis@gmail.com}}{\textbf{(+86) 188-130-18216}}{\textbf{应聘职位:视觉算法研发工程师}}
% {E-mail}{mobilephone}
%\basicContactInfo{\textbf{zhangxiaoya1989@gmail.com}}{\textbf{(+86) 188-130-18216}}
%\section{\faGraduationCap\  基本信息}
%\begin{itemize}[parsep=0.5ex]
%%\item \textbf{个人信息:} 男/硕士/智能信息技术北京市重点实验室
%\item \textbf{研究方向:} 图像与视频处理( Visual Tracking, Image Super Resolution),机器学习
%\item \textbf{期望职位:C++研发工程师}
\section{\faGraduationCap\  教育背景}
\datedsubsection{\textbf{北京理工大学(计算机科学与技术)2013年9月-2016年1月} }{硕士(top20\%)}
%\textit{在读硕士研究生}\ 计算机科学与技术, 预计 2016 年 3 月毕业
\datedsubsection{\textbf{青岛科技大学(计算机科学与技术)2008年9月-2012年7月} }{学士(top  3\%)}
%\textit{学士}\ 计算机科学与技术

\section{\faUsers\ 工作/项目经历}

\datedsubsection{\textbf{中科院自动化所(实习)} 北京}{2016年1月 -- 2016年4月}
\role{机器人平台组机器学习及机器视觉助理工程师}

\begin{itemize}
  \item 基于 Kinect 深度视觉信息的双臂机器人六自由度机械臂转动算法优化,提高算法精度,增强机械臂抓取羽毛球的鲁棒性;
  \item 改进基于稀疏相关模型(Sparsity-based Collaborative Model,SCM)目标跟踪算法,并应用于室内机器人跟踪模块,改善机器人跟踪特定目标的准确性。
  \item 研究PTAM,ORB-SLAM等单目视觉SLAM在室内机器人路径规划方面的应用。
  \item 基于ORB-SLAM,与室内的研究人员完成demo测试。
\end{itemize}

\datedsubsection{\textbf{北京源仪迅驰科技有限公司(工作)} 北京}{2017年3月 -- 至今}
\role{算法工程师}{}
C++ / Python / CUDA / OpenCV / YOLO / TensorFlow

\role{\textbf{项目一:基于深度学习的室外场景可疑目标检测预研项目}}

基于现有的光学设备,增强监控设备成像效果,去除雾霾、雨雪,对特定目标进行超分辨率增强,基于SSD的目标检测算法自动识别图像中的目标,目前已完成图像增强和图像超分辨率增强算法研究与实现。
\begin{itemize}
  \item 负责研究并改进现有的图像超分辨图像增强算法,研究在野外自然场景中图像超分辨算法实施可行性及具体实施方案;
  \item 负责基于现有的测试数据,利用OpenCV和Eigen开源第三方库,根据设计方案实现图像超分辨算法;
  \item 负责研究并总结基于暗通道的图像去雨、雪的图像增强算法,分析算法实施可行性并完成具体实施方案;
  \item 负责研究 基于SSD深度学习模型的目标检测算法,利用以后的数据训练模型,完成预研项目第一阶段全部工作。
\end{itemize}


\role {\textbf{项目二:基于图像处理的无人机等天空背景弱小检测与跟踪}}

针对无人机的飞行会对民航机场等重要场所造成安全威胁应用场景,负责基于计算机视觉、机器学习、深度学习等技术,利用高分辨率相机和红外相机采集的图像数据,设计并实现算法检测并跟踪进入管制区域的无人机,目前完成第一段算法设计与实现,现场测试结果表明识别率远超其他算法和竞标部门算法效果。
\begin{itemize}
  \item 负责整体设计基于超像素分割的单帧红外图像弱小目标检测算法,相对于传统检测算法,提升目标检测率,满足多种复杂天气实际需求。
  \item 基于CUDA和Eigen,负责实现实时目标检测和跟踪算法;
  \item 基于YOLO深度学习模型,训练弱小无人机目标检测模型,实现对天空域无人机等目标检测算法;
  \item 整合红外图像基于过分割的目标检测算法和可见光图像基于YOLO深度学习模型的目标检测算法,完成整体实时弱小目标检测和跟踪算法设计与实现。
  \item 设计基于Particle Filter算法,设计并实现改进表观特征的弱小目标跟踪算法。
\end{itemize}

\datedsubsection{\textbf{ThoughtWorks(工作)} 北京}{2016年5月 -- 2017年2月}
\role{全栈工程师}{}
.NET / Angular JS / PowerShell

项目:某四大会计事务所之一的系统开发
\begin{itemize}
  \item 使用测试驱动开发方法,维护原有的系统功能,增加新业务开发;
  \item 与业务分析师BA、测试工程师QA一起分解业务、分析业务,并与客户沟通业务逻辑;
  \item 基于Sonar和PowerShell脚本,在CI基础上搭建静态代码扫描工具;
\end{itemize}

\datedsubsection{\textbf{ThoughtWorks(其他)} 印度.浦那}{2016年8月 -- 2016年9月}
\role{全栈工程师}{}
Java / Angular JS

项目: Thoughtworks University
\begin{itemize}
  \item 参加Thoughtworks University,与不同国家的同事交流、合作完整基础训练;
  \item 与不同国家的同事组成开发小组、完成一个月的短期项目开发,包括与客户沟通、敏捷开发;
  \item 参加当地NGO的慈善活动;
\end{itemize}


\section{\faUsers\ 科研经历}

\datedsubsection{\textbf{基于字典学习的鲁棒性视觉跟踪算法的研究}}{2014年06月 -- 2015年10月}
%\role{国家自然科学基金}{研究生课题}
\begin{onehalfspacing}
基于机器学习方法,改进现有的基于字典学习目标跟踪方法,提出并实现鲁棒性更强的目标跟踪算法
\begin{itemize}
  \item 研究基于机器学习的目标跟踪算法、贝叶斯推理和粒子滤波理论
  \item 多字典学习,动态权衡给定目标特征与最新目标特征,在目标发生形变场景中的准确率
%  \item xxx (尽量使用量化的客观结果)
\end{itemize}
\end{onehalfspacing}

\datedsubsection{\textbf{基于BP神经网络的图像超分辨率算法的研究}}{2013年09月 -- 2014年06月}
%\role{国家自然科学基金}{研究生课题}
\begin{onehalfspacing}
研究基于 BP 神经网络多帧图像超分辨率算法,基于受限玻尔兹曼机的单帧图像超分辨
\begin{itemize}
  \item 总结已有的基于 BP 神经网络的图像超分辨率算法,实现基于受限玻尔兹曼机的单帧图像超分辨率算法;研究并改进基于 position patch 的图像超分辨率算法
%  \item 提出并实现基于修正字典增强鲁棒性的目标跟踪算法、发表一篇学术论文
%  \item xxx (尽量使用量化的客观结果)
\end{itemize}
\end{onehalfspacing}

%\section{\faHeartO\ 获奖情况}
%\begin{itemize}[parsep=0.5ex]
%  \item 荣誉称号: 山东省优秀毕业生、青岛科技大学优秀毕业生
%  \item 奖学金: 国家励志奖学金(2 次)、一等奖学金(4 次)、二等奖学金(3 次)
%\end{itemize}
%\datedline{荣誉称号:}{\textit{第一名}, xxx 比赛}{2013 年6 月}
%\datedline{其他奖项}{2015}

\section{\faCogs\ IT 技能}
% increase linespacing [parsep=0.5ex]
\begin{itemize}[parsep=0.5ex]
  \item 编程语言: C++ > C > Python > C\# > Lisp
  \item 深度技能: 图像处理/机器学习/深度学习/ CUDA / vSLAM
  \item 其他技能: $Git\& Github$ / 快速阅读开源项目源码 / 英语 - 能够熟练阅读文献,口语交流
\end{itemize}

\section{\faInfo\ 其他信息}
% increase linespacing [parsep=0.5ex]
\begin{itemize}[parsep=0.5ex]
  \item 技术博客: http://zhangxiaoya.github.io
  \item GitHub: https://github.com/zhangxiaoya
  \item 阅读开源项目:ORB-SLAM,DWoB、VINS-Mono
\end{itemize}

\end{document}
